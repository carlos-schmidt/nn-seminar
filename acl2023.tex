% This must be in the first 5 lines to tell arXiv to use pdfLaTeX, which is strongly recommended.
\pdfoutput=1
% In particular, the hyperref package requires pdfLaTeX in order to break URLs across lines.

\documentclass[11pt]{article}
% Remove the "review" option to generate the final version.
\usepackage[review]{ACL2023}
\usepackage{times}
\usepackage{latexsym}
\usepackage[T1]{fontenc}
% See https://www.latex-project.org/help/documentation/encguide.pdf for other character sets
\usepackage[utf8]{inputenc}
\usepackage{microtype}
\usepackage{inconsolata}

\newcommand{\enquote}[1]{``#1''}

% If the title and author information does not fit in the area allocated, uncomment the following
%
%\setlength\titlebox{<dim>}
%
% and set <dim> to something 5cm or larger.

\title{A Review on Multi-modal Speech Representation Learning}

\author{Carlos Schmidt \\\texttt{uuwss@student.kit.edu}\\}

\begin{document}
\maketitle
\begin{abstract}
This document is a supplement to the general instructions for *ACL authors. It contains instructions for using the \LaTeX{} style file for ACL 2023.
The document itself conforms to its own specifications, and is, therefore, an example of what your manuscript should look like.
These instructions should be used both for papers submitted for review and for final versions of accepted papers.
\end{abstract}

\section{Introduction}

What are speech representations? why would one want to learn them? connections to text and their successes (BERT, ChatGPT)

Motivate speech representation learning then show structure of document. This paper partly references a review on Self-Supervised Speech Representation Learning\cite{srl-review}

\subsection{History of Speech Representation Learning}

Show first approaches (?) 

\subsection{Self-Supervised Learning for Speech Representation Models}

1. Motivation for self supervised learning / What is self supervised learning in general. Where does it maybe originate from?\\
With the rise of deep learning, the use of labeled data to train models capable of \dots
HOW TO INTRO?~WITHOUT USING WORDS OF~\cite{srl-review}?

One example of labeled speech data is paired audio-text data, which consists of pairs of voice tracks and corresponding text segments. These kind of data can for example be used to train end-to-end automatic speech representation (ASR) models.\\Human-labeled speech data is expensive and generally of limited supply, especially so for languages with comparatively few speakers in the world.\\This is why methods using only unlabeled speech data were developed and are since being used to tackle many natural language processing tasks such as the aforementioned ASR task~\cite{unsupervised_learning}.

Self-supervised learning comprises of \enquote{techniques that utilize information extracted from the input data itself as the label to learn representations useful
for downstream tasks.}~\cite{srl-review}\\
2. Bridge to speech/natural language processing\\
3. Pre-training. Generative Learning. Contrastive Learning. Predictive Learning.


\section{Single-mode Speech Representation Models}

\subsection{wav2vec2.0}

\subsection{HuBERT}

\section{Multi-modal Speech Representation Models}

This section presents three examples of speech representation learning models each leveraging a different set of input channels. First, we will look at SpeechT5~\cite{speecht5} which is using the additional text modality for learning speech representations. Next, we will see AV-HuBERT~\cite{AV_HuBERT}, which is an extension of the single-mode model HuBERT~\cite{} and benefits from the audio modality as well as the video modality. Lastly, VAT-LM~\cite{} will be covered, leveraging all three mentioned modalities: audio, video and text.

\subsection{SpeechT5}

The SpeechT5 framework, first introduced in 2021 by Microsoft~\cite{speecht5}, is an expansion of the text-only T5 framework (Text-to-Text Transfer Transformer,~\citeposs{t5}). SpeechT5 is\dots

\subsubsection{Model Architecture}
\subsubsection{Learning (?)}
\subsubsection{Performance Discussion}

Here, we show that SpeechT5 outperforms single-mode SRL models, and also the benefit of using multiple modalities by presenting the ablation studies of the original paper.~See Table~\ref{table:speecht5_performance}. Also, here is a section of the appendix containing more information:~\ref{sec:appendix}.

\begin{table}
    \centering
    \begin{tabular}{lc}
    \hline
    \textbf{Model} & \textbf{WER~(\%)}\\
    \hline
    wav2vec2.0 & 1.8\\
    SpeechT5 & 1.5\\
    \hline
    \end{tabular}
    \caption{\label{table:speecht5_performance}ASR Performance.}
\end{table}

\subsection{AV-HuBERT}

AV-HuBERT Stuffs

\subsubsection{Model Architecture}
\subsubsection{Learning (?)}
\subsubsection{Performance Discussion}

\subsection{VAT-LM}

VAT-LM Stuffs

\subsubsection{Model Architecture}
\subsubsection{Learning (?)}
\subsubsection{Performance Discussion}


\section{Discussion}
Discussion on how well multi-modal models improve the field of SRL.\@

\section*{Ethics Statement}
Scientific work published at ACL 2023 must comply with the ACL Ethics Policy.\footnote{\url{https://www.aclweb.org/portal/content/acl-code-ethics}} We encourage all authors to include an explicit ethics statement on the broader impact of the work, or other ethical considerations after the conclusion but before the references. The ethics statement will not count toward the page limit (8 pages for long, 4 pages for short papers).

\section*{Acknowledgements}
Maybe acknowledge the review~\cite{srl-review} or smth.

% Entries for the entire Anthology, followed by custom entries
\bibliography{anthology,custom}
\bibliographystyle{acl_natbib}

\appendix

\section{Example Appendix}\label{sec:appendix}

This is a section in the appendix.

\end{document}
